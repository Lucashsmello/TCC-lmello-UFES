
\newcommand{\NbasesV}{\textit{N}}
\newcommand{\Nbases}{123}
\newcommand{\Nml}{6}
\newcommand{\NmlT}{4}
\newcommand{\NmlA}{2}
\newcommand{\Ncb}{4}
\newcommand{\MML}{método multirrótulo}
\newcommand{\MMLs}{métodos multirrótulo}
\newcommand{\MRLM}{Método Revolucionário de Lucas Mello}
\newcommand{\MRLMa}{MRLM}

\chapter{Introdução}
\begin{verbatim}
 -Falar um pouco sobre aprendizado de máquina.
 -Falar sobre Classificação multiclasse
 -Apresentar Classificação multirrótulo e falar como é mais tenso que a multiclasse, dar exemplos etc.
 -Falar um pouco do estado da arte da Classificacao multirrotulo.
 
\end{verbatim}



\section{Motivações}
% Análises dos métodos multirrótulo trazem os seguintes benefícios:
O melhor entendimento do funcionamento dos métodos multirrótulo permite:
\begin{itemize}
 \item descobrir atributos destes que se alterados, aproveitados
 e/ou combinados podem acarretar na criação de novos métodos e/ou na melhora dos existentes.
 \item prever, com uma certa taxa de erro, seus desempenhos, o que facilita o uso mais inteligente dos métodos
 sem precisar utilizar muito esforço computacional devido a testes.
 \item reforçar ou contrariar as conclusões já estabelecidas dos métodos, uma vez que a maioria delas são 
 baseadas em testes experimentais.
\end{itemize}

\section{Objetivos}
O objetivo geral deste trabalho é analisar e comparar métodos multirrótulos distintos.
Mais formalmente, a análise deve implicar em conclusões matemáticas ou estatísticas sobre o desempenho dos métodos multirrótulo.
Objetivo geral pode ser detalhado nos seguintes objetivos específicos:
\begin{itemize}
 \item Comparação estatística e análise crítica dos métodos multirrótulos;
 \item Implementação dos métodos multirrótulos em uma biblioteca que integra técnicas de reconhecimento de padrões.
\end{itemize}


% \section{Contribuições}
\section{Estrutura do Trabalho}

\chapter{Classificação multirrótulo}

\section{Enunciado do problema}
\begin{verbatim}
ESBOÇO:
-Definição formal.
-um classificador multirrótulo mapeia do espaço de características
para um vetor de valores reais entre 0 e 1 
de tamanho igual ao número de rótulos.
-falar da complexidade do problema
-falar sobre lidar o problema como probabilidade condicional dado o rótulo e etc...
    P(l_1 | x,l_2)
\end{verbatim}



\section{Avaliação de Desempenho}
\begin{verbatim}
-Apresentar matematicamente o que pode indicar correlação entre rótulos e outras coisas
conforme o artigo: "Bayes Optimal Multilabel Classification via Probabilistic Classifier Chains".
-Falar que é diferente do multiclasse
\end{verbatim}
\subsection{Métricas}
\label{sec:metrics}
\subsubsection{Accuracy}
\subsubsection{Hamming-Loss}
\subsubsection{Subset Accuracy}
\subsubsection{Ranking Loss}
\subsection{Modelo de Avaliação}
\label{sec:modelav}

\begin{verbatim}
Falar do Holdout,Cross-validation estratificado.
Falar sobre bases separadas de treino,teste e validação.
\end{verbatim}

\chapter{Métodos Multirrótulos}
\section{Transformação do Problema}
\subsection{Relevância Binária - BR}
\subsection{Classifier Chain}
\subsection{Ensemble of Classifier Chain}
\subsection{Probabilistic Classifier Chain}
\subsection{LEAD}
\subsection{MAIS MÉTODOS...}

\section{Adaptação de classificadores}
\subsection{ML-KNN}
\subsection{Rede Neural Artificial}
\subsection{C4.5 multirrótulo}
\subsection{CRankSVM}
\subsection{MAIS...}


\chapter{Super Novo Método Revolucionário*}
A proposta de \MRLM é fundamentada no \MML~\textit{dependent binary relevance} (DBR) \cite{dbr2014} portanto,
esse capítulo é iniciado com uma seção explicando o funcionamento do DBR, facilitando o entendimento do \MRLMa~na seção seguinte.

\section{Dependent Binary Relevance - DBR}
\section{Algoritmo \MRLM~-~\MRLMa}
Como foi dito anteriormente, o \MRLM~é baseado no DBR. 
% Ambos se baseiam na expansão do espaço de características
% com características que representam a estimativas dos rótulos como forma de explorar
% a correlação entre rótulos. Isso é feito usando a predição do BR no espaço de características original.
% Ou seja, 
Ambos dependem da hipótese de que as estimativas dos rótulos em $Y$ por um classificador multirrótulo $c_0$
são boas características para aprimorar as estimativas dos mesmos rótulos por um novo classificador $c_1$
e que quanto melhor forem as estimativas dos rótulos por $c_0$, melhores são as de $c_1$.
E ainda ambos usam o BR como classificador multirrótulo base, que servirá para realizar as primeiras estimativas dos 
rótulos.
 No entanto, o \MRLMa, ao invés de usar apenas dois classificadores, como o DBR, 
 usa $n$ classificadores multirrótulo $c_0,c_1,...,c_{n-1}$ organizados em uma cadeia de tal forma que
 as predições do classificador $c_{i-1}$ , para $1\leq i<n$, são usados como características para o classificador $c_{i}$. 
 Dessa forma, o classificador $c_{i}$ será baseado em estimativas melhores de rótulos do que seus antecessores $c_{i-1},c_{i-2},...,c_0$ e ainda
 o classificador $c_{i}$ será capaz de aprender sobre os erros e acertos do classificador $c_{i-1}$ e alcançar resultados melhores.
 
 Adicionalmente, o \MRLMa~assume que as características $\hat{y}_1,\hat{y}_2,...,\hat{y}_l$, estimados por um classificador multirrótulo,
 são suficientes ou quase suficientes para predizer corretamente os valores reais dos rótulos em $Y$.
 Baseado nessa suposição, o \MRLMa~substitue todas as características do espaço original pelas características $\hat{y}_1,\hat{y}_2,...,\hat{y}_l$,
 para todos os seus classificadores, exceto para o classificador base $c_0$, uma vez que não existe ainda uma estimativa dos rótulos a usar.
 Isso reduz o espaço de características o que diminui o custo computacional dos algoritmos.
 Por outro lado, essa suposição nem sempre se mantém, podendo prejudicar a qualidade de predição dos algoritmos, mas espera-se que em média se mantenha.
 
 Formalmente, a estrutura do \MRLMa~é organizado da seguinte forma:
 \begin{itemize}
  \item Para um valor pré-definido de n tal que $n\in \mathbb{N}, n>0$, o algoritmo é composto de
  $n$ classsificadores BR que formam o conjunto $C=\{c_0,c_1,...,c_{n-1}\}$. 
  \item Os classificadores em $C$ estão ordenados e dispostos em uma cadeia de tal forma que o 
  classificador $c_i$ está ligado a saída (estimativas dos rótulos) do classificador $c_{i-1}$ e apenas a ele.
  \item Cada classificador multirrótulo $c_i$ é composto de $l$ classificadores binários $h_{i1},h_{i2},h_{i3},...,h_{il}$
  dispostos em uma cadeia.
  
  
  Veja a figura \ref{fig:mrlm_struct} para visualizar a estrutura do \MRLMa.
  
  
  
%   de tal forma que , para um $j\in \mathbb{N}, 0<j\leq l$, o $h_{ij}$ está ligado a $h_{i(j-1)}$ e somente a ele.
  
 \end{itemize}

 
 
 \subsection{Fase de Treinamento}
%  Dado uma base de treino $D$ composta de $n$ instâncias 
 
 Formalmente, o \MRLMa~funciona da seguinte forma. Os classificadores em $c_0,c_1,...,c_{n-1}$ são representados
 pelas funções $h_0,h_1,...,h_{n-1}$:
 \begin{equation}
 \begin{split}
    & h_0 : X \rightarrow [0,1]^l \\
    & h_i : [0,1]^l \rightarrow [0,1]^l | i=1,...,n-1
%   h_0 : X \rightarrow \mathbb{R}^l &\\
%   h_i : \mathbb{R}^l \rightarrow \mathbb{R}^l & | i=1,...,n-1
  \end{split}
 \end{equation}
 
 onde a imagem de cada função $h_i$ é um vetor $(\hat{y}_1,\hat{y}_2,...,\hat{y}_l)$ de tamanho $l$ onde cada elemento
 $\hat{y}_i$ é um valor entre 0 e 1 que representa a pertinência de uma instância $t$ para o rótulo $i$ ou a probabilidade da instância $t$ pertencer
 ao rótulo $i$.
 
 Primeiramente, treinamos o classificador BR $c_0$ no espaço de
 características original $X$, resultando na construção da função $h_0(\bf{x})$:
 \begin{equation}
  h_0 : X \rightarrow \{0,1\}^l\.{z} 
 \end{equation}
 Em seguida, aplica-se a função $h_0$ sobre todas as instâncias da base de treino, resultando nas saídas
 $h_0(x_1), h_0(x_2),...,h_0(x_n)$. A partir daí as instâncias são transformadas e mapeadas para um novo espaço de
 características, criando a nova base de dados $D'$:
 \begin{equation}
  D'=\{(h_0(x_i),y_i)|i=1,...,n\}
 \end{equation}

 
%  \begin{equation}
% \label{eq:funcT}
%   T : X \times Y \rightarrow \hat{Y} \times Y,~T(x,y)=(h_0(x),y)
%  \end{equation}
%  onde $\hat{Y}$ é composto por instâncias da forma:
%  \begin{equation}
%   x'=(h_0(\bf{x}))
%  \end{equation}


 
 \subsection{Fase de Predição}
 
%  A ideia por detrás de suposição é que o classificador $c_i$ já foi capaz de extrair todas as informações dispo
 

% No entanto,
% o \MRLMa~depende de duas novas hipóteses. A primeira é que 

% A primeira é que as estimativas dos rótulos em $Y-\{y_i\}$ podem ser usados para estimar
% o rótulo $y_i$ e quanto melhor forem as estimativas dos rótulos em $Y-\{y_i\}$, melhor é a de $y_i$. 
% A idéia de que ao supor que essa hipótese é
% verdade, o 
% A segunda é que a estimativa de um rótulo
%   \begin{enumerate}
%    \item 
%    
%    
%    Em outras palavras, 
%    
%    
%    A estimativa de um rótulo $y_i$ para uma instância qualquer pode ser feita, com uma certa taxa de erro $E$~
%    usando apenas as estimativas marginais de todos os rótulos, inclusive $y_i$, de tal forma que $E<e_i$, onde $e_i$ é taxa de erro
%    para estimativa marginal do rótulo .
%   \end{enumerate}

 



\chapter{Avaliação e Análise Experimental}
\section{Método de Comparação}

A comparação e estudo dos métodos é feita considerando o custo computacional e a qualidade de predição.
A qualidade de predição é estimada pelo método de avaliação por validação cruzada \ref{sec:modelav} com 10 \textit{folds}
sobre as \Nbases~bases de dados, todas apresentadas pela tabela \ref{tab:datas}.
Todas as bases de dados foram obtidas de repositórios públicos pelo seguinte
endereço virtual @@@.
Para quantificar a qualidade de predição, foram escolhidas as seguintes métricas, todas detalhadas na seção \ref{sec:metrics}:
% Hamming Loss, Accuracy, Subset Accuracy, Ranking Loss, ...@@@.

% As métrica escolhidas e os motivos pela qual foram escolhidas são os seguintes:
\begin{itemize}
 \item Subset Accuracy, pois é rígida @@@ etc...
 \item @@@
 \item @@@
 \item Tempo computacional, pois é @@@ etc...
\end{itemize}

É insteressante mostrar os resultados experimentais usando diferentes métricas, pois cada uma captura
um aspecto diferente da classificação.
No entanto, uma vez que é impossível obter uma única ordenação dos métodos
usando múltiplos critérios,
% e isso é necessário para
% observar a característica de rankeamento constante, citada na seção anterior,
optou-se por usar a da média dos ranking das métricas como
o único critério de ordenação ao levantar pontos conclusivos sobre os métodos. 

Num total, foram comparados \Nml~\MML, \NmlT~da categoria de transformação e \NmlA~de adaptação @@@.
A comparação é antecedida por um estudo dos parâmetros dos métodos
cujo objetivo principal é encontrar os parâmetros que devem ser usadas na comparação dos métodos. Objetivos secundários são
espeficicados nas seções \ref{sec:study1} e \ref{sec:study2}, especialmente dedicadas para esse estudo.
O objetivo principal é alcançado ao analisar a alteração do comportamento dos métodos quando seus parâmetros são alterados.
% , por exemplo,
% se são muito sensíveis aos seus parâmetros ou não. 
O estudo dos parâmetros foi dividido em dois subestudos:
\begin{enumerate}
 \item estudo individual, que tem por objetivo analisar os parâmetros únicos de cada método individualmente.
  \item estudo de grupo, que tem por objetivo analisar os parâmetros que são idênticos para um grupo de métodos.
  Nesse caso o parâmetro que representa a escolha do classificador binário base para os método baseados em transformação é
  analisado. 
\end{enumerate}
 No estudo de grupo há duas características principais que se desejam observar:
  
  \begin{enumerate}
  \item a variação do ranking dos métodos segundo um critério de qualidade ao variar o classificador base
  para uma base específica.
  \item a variação do ranking do classificador base segundo um critério de qualidade ao variar o método multirrotulo.
%    \item rankeamento constante, que acontece se a ordem dos métodos permanece a mesma independente
%    do valor do parâmetro quando ordenados por um critério de qualidade a uma base específica.
%    \item parâmetro ótimo, que é o classificador base que causa o melhor desempenho no \MMLs em uma base específica.
%    Caso o rankeamento seja inconstante, será um parâmetro ótimo para cada \MML.
  \end{enumerate}

  Para mais detalhes sobre os motivos desse estudo bem como seus resultados e conclusões na seção \ref{sec:study1}.
  

\section{Estudo prévio - NOME MELHOR DPOIS @@@}
\label{sec:study1}
\begin{verbatim}
 -Listar métodos.
 -Falar como vai ser comparado cada método (teste estatístico, rankeamento, métrica boa) para comprovar a hipotese.
 -Apresentar resultados. (experimentos,tabela, gráficos...)
 -Conclusão sobre os resultados
\end{verbatim}

Nessa seção os \NmlT~\MMLs~de transformação são estudados com relação a suas qualidades. (@@@ listar os métodos de transfomação.)
A escolha do classificador base para cada um dos métodos foi reestrito para \Ncb~classificadores: @@@.
Esse subestudo tem como objetivo principal encontrar os classificadores $c_1^*,c_2^*,c_3^*,c_4^*,c_5^*$ que serão utilizados pelos
métodos $T_1,T_2,T_3,T_4,T_5$.

% Como forma de comprovar a hipótese 1 aplicou-se o teste estatístico @@@.
% Assim, a
A comparação é realizada pelo rankeamento dos métodos, ou seja, os \MMLs~de transformação são submetidos a testes experimentais
e ordenados segundo seus desempenhos de tal forma que para cada métrica e classificador base,
é atribuído ao \MML~um valor correspondente a sua ordem, entre 1 e \NmlT. A ordenação por uma métrica
é independente da ordenação da outra. 
Os resultados experimentais são apresentados nas tabelas \ref{tab:study1-R-emotions},\ref{tab:study1-R-birds},..., uma para cada métrica utilizada.




\chapter{Conclusão}